\documentclass[10pt]{article}

% Manage page layout
\usepackage[margin=2.5cm, includefoot, footskip=30pt]{geometry}
\pagestyle{plain}
\setlength{\parindent}{0em}
\setlength{\parskip}{1em}
\renewcommand{\baselinestretch}{1}

\usepackage{blkarray}
\usepackage{multirow}
\usepackage{amsmath}

\title{\textbf{Week 2.} Static games with complete information I: Elimination of dominated strategies}
\date{}

\begin{document}
\maketitle
\vspace{-1cm}

\subsection*{Exercise 1: Cournot Duopoly}
There are two firms that produce a given good for a given market. Each firm needs
to decide how much of the good they want to produce (which can be a continuous quantity between 0 and \(\infty\)).

Producing one unit of the good comes at a cost of \(c > 0\). The price for which the
good can be sold is a decreasing function of the total amount of good produced,
\(p = a - b x\), where \(x\) is the total amount of the good produced by the
two firms.

\textbf{Translate this into a game, who are the players, what are the actions, what are the payoffs?}

\underline{\textbf{Bonus question:}} Are there any dominated actions?


\subsection*{Exercise 2: Symmetric games}
In the lecture we have defined for a 2-players game with finitely many actions
what it means to be symmetric. Basically, it means that the players' positions do not
mean anything. You could switch the roles of the player 1 and 2, and the payoffs would
remain the same.

\textbf{Can you come up with a definition of what it means that a \(n-\)players game is symmetric?}

[Hint: For this it might be useful to use the concept of a permutation. A permutation
is a bijective map \(\tau: \{1, \dots, n\} \rightarrow \{1, \dots, n\}\). Any re-ordering
of players can thus be described as a permutation.]


\subsection*{Exercise 3: Dominated strategies}
\textbf{Show that in the following game the pure strategy \(M\) is dominated.}

\[
\begin{blockarray}{ccc}
& L & R \\
\begin{block}{c(cc)}
    U & (3, 0) & (0, 0) \\
    M & (1, 1) & (1, 1) \\
    D & (0, 0) & (3, 0) \\
\end{block}
\end{blockarray}
\]

\textbf{Show that it is no longer dominated if all ``1''s are changed to ``2''s.}

\subsection*{Exercise 4: Iterated elimination of dominated strategies I}

\textbf{Construct a 2-player game such that:}

\begin{enumerate}
    \item Both players have 3 actions.
    \item The game cannot be solved by elimination of dominated strategies.
    \item The game can be solved by elimination of dominated strategies.
\end{enumerate}

\subsection*{Exercise 5: Iterated elimination of dominated strategies II}
Consider the following 3-players game. Here the first player chooses a row,
the second player chooses a column, and the third player chooses a matrix.

\textbf{Can you solve this game using iterated elimination of strategies?}

\begin{equation*}
\begin{blockarray}{ccc}
    & \BAmulticolumn{2}{c}{\text{Matrix } 1} \\ [1em]
    & \text{Col } 1 & \text{Col } 2 \\
    \begin{block}{c(cc)}
        \text{Row } 1 & (2, 1, 2) & (3, 2, 3) \\
        \text{Row } 2 & (0, 4, 0) & (1, 0, 0) \\
    \end{block}
\end{blockarray}\qquad
%
\begin{blockarray}{ccc}
    & \BAmulticolumn{2}{c}{\text{Matrix } 2} \\ [1em]
    & \text{Col } 1 & \text{Col } 2 \\
    \begin{block}{c(cc)}
        \text{Row } 1 & (1, -1, 4) & (2, 1, 4) \\
        \text{Row } 2 & (-1, 4, 0) & (0, 0, 3) \\
    \end{block}
\end{blockarray}
\end{equation*}

\end{document}

