\documentclass[10pt]{article}

% Manage page layout
\usepackage[margin=2.5cm, includefoot, footskip=30pt]{geometry}
\pagestyle{plain}
\setlength{\parindent}{0em}
\setlength{\parskip}{1em}
\renewcommand{\baselinestretch}{1}

\usepackage{blkarray}
\usepackage{multirow}
\usepackage{amsmath}
\usepackage{eurosym}
\usepackage{enumerate}

\title{\textbf{Week 6.} Sequential games with complete information II: Subgame perfection and repeated games}
\date{}

\begin{document}
\maketitle
\vspace{-1cm}

\subsection*{Exercise 1: Battle of the sexes with an outside option}

The battle of sexes is a game where two people would prefer to do something
together, but each person likes a different activity best. The payoff matrix is,

\begin{equation*}
    \begin{blockarray}{ccc}
        & a_1 & a_2 \\
        \begin{block}{c(cc)}
            a_1 & (3, 1) & (0, 0) \\
            a_1 & (0, 0) & (1, 3) \\
        \end{block}
    \end{blockarray}
\end{equation*}

Now suppose that before playing this game, player 1 can choose whether to play
this game or to exit. If player 1 exits, both players obtain a payoff of 2.

\textbf{Show that the battle of sexes with an outside option has two pure
subgame perfect equilibria:}

\begin{enumerate}[(i)]
    \item Player 1 plays, and both players choose activity \(a_1\).
    \item Player 1 exits, because if they were to play they would both choose
    activity \(a_2\).
\end{enumerate}

\underline{\textbf{Bonus question:}} Can you give a compelling argument why
player 1 may be able to undermine the second equilibrium?

\subsection*{Exercise 2: Matching pennies with outside option}

Consider the matching pennies games with the following matrix (Example 3.12),

\begin{equation*}
    \begin{blockarray}{ccc}
        & \text{Left} & \text{Right} \\
        \begin{block}{c(cc)}
            \text{Up} &    (0.8, 0.4) & (0.4, 0.8) \\
            \text{Down} & (0.4, 0.8) & (0.8, 0.4) \\
        \end{block}
    \end{blockarray}\qquad
\end{equation*}

and assume that before playing this game player 1 can choose whether to play
this game or to exit.

\begin{itemize}
    \item What are the players' possible pure strategies?
    \item Write the games as a payoff matrix. Is there a pure Nash equilibrium?
    If yes why is the mixed equilibrium derived in Example 3.12 more
    compelling?
\end{itemize}

\subsection*{Exercise 3: A finite repeated games}

Consider the stag game with the payoff matrix,

\begin{equation*}
    \begin{blockarray}{cccc}
        & C & D_1 & D_2 \\
        \begin{block}{c(ccc)}
            C   & (3, 3)   & (0, 4)   & (-12, 0) \\
            D_1 & (4, 0)   & (1, 1)   & (-10, 0) \\
            D_2 & (0, -12) & (0, -10) & (-5, -5) \\
        \end{block}
    \end{blockarray}\qquad
\end{equation*}

\textbf{Show that:}

\begin{enumerate}[(i)]
    \item If the game is only player once, there is \underline{no} Nash equilibrium
    in which \(C\) is player with a positive probability.
    \item If the game is played twice, there is a subgame perfect equilibrium
    in which \(C\) is player in the first round.
\end{enumerate}


[Hint: Consider the strategy: Play \(C\) in the first round. If both players
played \(C\) in the first round play \(D_1\) in the second round, otherwise play
\(D_2\).]

\subsection*{Bonus Exercise 1}

Consider the infinitely repeated prisoner's dilemma with payoffs,

\begin{equation*}
    \begin{blockarray}{ccc}
        & C & D \\
        \begin{block}{c(cc)}
            C & (3, 3) & (0, 4) \\
            D & (4, 1) & (1, 1) \\
        \end{block}
    \end{blockarray}\qquad
\end{equation*}


\textbf{Prove that there are sequences of actions such that players' average payoff,}

\[\frac{1}{T+1} \sum\limits_{t=0}^{T} u^{(i)}(a_t)\]

\textbf{does not converge as \(T \rightarrow \infty\).}

[Hint: Consider the case that both players first play \(C\) for one round. Then they
play \(D\) for 2 rounds. Then they play \(C\) for four rounds. Then they play
\(D\) for 8 rounds.]

\end{document}

