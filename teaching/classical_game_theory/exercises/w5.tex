\documentclass[10pt]{article}

% Manage page layout
\usepackage[margin=2.5cm, includefoot, footskip=30pt]{geometry}
\pagestyle{plain}
\setlength{\parindent}{0em}
\setlength{\parskip}{1em}
\renewcommand{\baselinestretch}{1}

\usepackage{blkarray}
\usepackage{multirow}
\usepackage{amsmath}
\usepackage{eurosym}
\usepackage{enumerate}

\title{\textbf{Week 5.} Sequential games with complete information I: Backward Induction}
\date{}

\begin{document}
\maketitle


\subsection*{Exercise 1: Sequential prisoner's dilemma with remorse}

Revisit the prisoner's dilemma with remorse (Example 2.13 from the lecture). The
payoff matrix is:

\begin{equation*}
    \begin{blockarray}{ccc}
        & \text{Silence} & \text{Confess} \\
        \begin{block}{c(cc)}
            \text{Silence} & (3, 3) & (0, 0) \\
            \text{Confess} & (4, 0) & (1, 1) \\
        \end{block}
    \end{blockarray}
\end{equation*}

We already know that if both players move simultaneously, then the only Nash
equilibrium is (Confess, Confess).

\textbf{Now solve this game with backward induction, assuming that}

\begin{enumerate}
    \item the row player moves first,
    \item the column player moves first,
\end{enumerate}

\textbf{and interpret the result.}

\subsection*{Exercise 2: Ultimatum bargain in 2 rounds}

There is a good that is worth 1\euro~to the buyer and 0\euro~to the seller.
Sequence of moves:

\begin{enumerate}
    \item Seller names a price \(p \in \{0.01, 0.02, \dots, 0.99\}\).
    \item Buyer accepts or rejects the offer. If buyer accepts, the payoffs are
    \(p\) for the seller, \(1 - p\) for the buyer and the game is over.
    \item Otherwise, the buyer names a price \(p \in \{0.01, 0.02, \dots,
    0.99\}\).
    \item Seller accepts or rejects the offer. If the seller accepts, the
    payoffs are \(p - \delta\) for the seller, and \(1 - p - \delta\) for the buyer
    where \(\delta=0.045\) reflects the cost of having to go through a long
    negotiation. If the seller rejects, the payoff of both players is
    \(-\delta\).
\end{enumerate}

\textbf{Solve by backward induction, and interpret the result.}

\subsection*{Exercise 3: Stackelberg Duopoly}

Similar to the Cournot duopoly game consider the following situation:

\begin{itemize}
    \item Players: \(N = \{\text{Firm } 1, \text{Firm } 2\}\)
    \item Actions: Amount of good produced, \(x^{(i)} \in [0, \infty)\) for \(i \in \{1, 2\}\)
    \item Payoffs: \(\pi^{(i)}(x^{(1)}, x^{(2)}) = [a - b (x^{(1)} + x^{(2)})] x^{(i)} - c x^{(i)}\)
\end{itemize}

However, now assume that Firm 1 decides first, and Firm 2 observes Firms 1's
decision before choosing an action.

\textbf{Solve the game by backward induction (for \(a=10, b=1, c=1\)) and
compare the result to the Nash equilibrium of the Cournot duopoly
(\(\hat x^{(1)} = \hat x ^{(2)}  = 3)\)}.

[Hint: First you should find the best response of Firm 2. That is, for any given
output \(x^{(1)}\) of Firm 1, compute how Firm 2 would react if it wants to maximize
its payoffs. Using this, compute how Firm 1 can maximize its own payoff when it
takes into account that Firm 2 will react according to its best response.]

\subsection*{Bonus Exercise 1: Properties of backward induction}

\textbf{
Prove that for any finite game with perfect information (i.e. in any game in which
backward induction can be applied) the solution defined by backward induction
is a Nash equilibrium.}


\end{document}

