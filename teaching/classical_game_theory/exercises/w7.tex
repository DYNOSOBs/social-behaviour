\documentclass[10pt]{article}

% Manage page layout
\usepackage[margin=2.5cm, includefoot, footskip=30pt]{geometry}
\pagestyle{plain}
\setlength{\parindent}{0em}
\setlength{\parskip}{1em}
\renewcommand{\baselinestretch}{1}

\usepackage{blkarray}
\usepackage{multirow}
\usepackage{amsmath}
\usepackage{eurosym}
\usepackage{enumerate}

\title{\textbf{Week 7.} Sequential games with complete information III: Repeated games}
\date{}

\begin{document}
\maketitle

\subsection*{Exercise 1: Win-Stay Lose-Shift}

The strategy Win-Stay Lose-Shift (WSLS) cooperates in the first round. In all
subsequent rounds, it cooperates if either both players cooperated in the
previous round, or if no one did. Otherwise it defects.

\textbf{Show that for the infinitely repeated prisoner's dilemma with stage game
payoffs}

\begin{equation*}
    \begin{blockarray}{ccc}
        & C & D \\
        \begin{block}{c(cc)}
            C & (3, 3) & (0, 4) \\
            D & (4, 0) & (1, 1) \\
        \end{block}
    \end{blockarray}
\end{equation*}

\textbf{the strategy profile (WSLS, SWLS) is a subgame perfect equilibrium if
\(\delta \geq \frac{1}{2}\).}

[Hint: Similarly to the examples covered in class, to prove that the strategy
profile (WSLS, WSLS) is a subgame perfect equilibrium we need to check different
cases. For case one consider a history \(h_t\) according to which either both
players cooperated in the previous round, or both players defected. For case two
consider that one player defected.]

\subsection*{Exercise 2: Mini-Max I}

Consider the matching pennies games

\begin{equation*}
    \begin{blockarray}{ccc}
        & \text{Left} & \text{Right} \\
        \begin{block}{c(cc)}
            \text{Up} &    (0.8, 0.4) & (0.4, 0.8) \\
            \text{Down} & (0.4, 0.8) & (0.8, 0.4) \\
        \end{block}
    \end{blockarray}\qquad
    \end{equation*}

Show that in the definition of minimax, it is important to allow for mixed
strategies of the opponent.

\textbf{Specifically show that:}

\begin{align*}
    \min\limits_{s^{(2)}} \max\limits_{s^{(1)}} u^{(1)}(s^{(1)}, s^{(2)}) & = 0.8, \text{ but} \\
    \min\limits_{\sigma^{(2)}} \max\limits_{s^{(1)}} u^{(1)}(s^{(1)}, \sigma^{(2)}) & = 0.6.
\end{align*}


\subsection*{Bonus 1: Mini-Max II}

Show that the minimax payoff of a player can be lower than what this player
could get in a Nash equilibrium. Specifically, consider the game

\begin{equation*}
    \begin{blockarray}{cccc}
        & \text{Left} & \text{Right} \\
        \begin{block}{c(ccc)}
            \text{Up}  & (-2, 2)   & (1, -2)  \\
            \text{Medium} & (1, -2)   & (-2, 2)  \\
            \text{Down} & (0, 1) & (0, 1) \\
        \end{block}
    \end{blockarray}\qquad
\end{equation*}

\begin{itemize}
    \item \textbf{Show that} the Nash equilibria of this game are of the form
    \begin{align*}
        \sigma^{(1)} & = (0, 0, 1) \\
        \sigma^{(2)} & = (q, 1- q) \text{ with } q \in \left[\frac{1}{3}, \frac{2}{3}\right],\\
    \end{align*}
    and the resulting payoffs are \(\hat{u}^{(1)} = 0, \hat{u}^{(2)} = 1\).
    \item \textbf{Show that} player 2's minimax payoff \(\underline{u}^{(2)} =
    \min\limits_{\sigma^{(2)}} \max\limits_{s^{(1)}} u^{(2)}(s^{(1)},
    \sigma^{(2)}) = 0 < \hat{u}^{(2)}\).
\end{itemize}

Now that you have shown that the minimax payoff of a player can be lower than
their Nash equilibrium payoff, \textbf{conclude that in repeated games with a
sufficient large \(\delta\), players may be worse off in equilibrium than in the
one shot game.}

\subsection*{Bonus Exercise 2: Folk Theorem}

Consider the battle of the sexes

\begin{equation*}
    \begin{blockarray}{ccc}
        & a_1 & a_2 \\
        \begin{block}{c(cc)}
            a_1 & (3, 1) & (0, 0) \\
            a_2 & (0, 0) & (1, 3) \\
        \end{block}
    \end{blockarray}
\end{equation*}

\textbf{What is the set of feasible and individually rational payoffs?}

\underline{Bonus:} Construct a strategy \(\hat{\sigma}\) for the repeated battle of sexes
that for a sufficiently large \(\delta\) satisfies the following 2 conditions.

\begin{enumerate}[(i)]
    \item When both player adopt the strategy, they obtain a payoff of approximately
    \(\pi^{(1)} = \pi^{(2)} = 2.\)
    \item \((\hat{\sigma}, \hat{\sigma})\) is a subgame perfect equilibrium.
    You do not need to show this rigorously, but give a convincing argument.
\end{enumerate}



\end{document}

