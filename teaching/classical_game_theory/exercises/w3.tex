\documentclass[10pt]{article}

% Manage page layout
\usepackage[margin=2.5cm, includefoot, footskip=30pt]{geometry}
\pagestyle{plain}
\setlength{\parindent}{0em}
\setlength{\parskip}{1em}
\renewcommand{\baselinestretch}{1}

\usepackage{blkarray}
\usepackage{multirow}
\usepackage{amsmath}
\usepackage{enumerate}

\title{\textbf{Week 3.} Static games with complete information II: Nash equilibrium}
\date{}

\begin{document}
\maketitle
\vspace{-1cm}

\subsection*{Exercise 1: Iterated elimination of dominated strategies I (previous exercise 4)}

\subsection*{Exercise 2: Iterated elimination of dominated strategies II (previous exercise 5)}

\subsection*{Exercise 3: Nash equilibrium vs dominance solvability}

\textbf{Prove the following statements:}

\begin{enumerate}[(i)]
    \item If a pure strategy \(S^{(i)}_{j}\) is dominated and 
    \(\sigma = (\sigma^{(1)}, \dots, \sigma^{(n)})\) is a Nash equilibrium, then
    \(\sigma^{(i)}_{j}=0\).
    \item If the game is dominance solvable such that the unique outcome of
    iterated elimination of dominated strategies is some pure strategy
    \(s=(s^{(1)}, \dots, s^{(n)})\), then \(s\) is a Nash equilibrium.
\end{enumerate}


[Hint: In each case prove by contradiction. For example, for (i) assume that these was a
Nash equilibrium with \(\sigma^{(i)}_{j}>0\),
and show that this would yield some contradiction.]

\subsection*{Exercise 4: Best responses}

Consider the stag hunt game:

\begin{equation*}
    \begin{blockarray}{cccc}
       & & \BAmulticolumn{2}{c}{\underline{\text{player 2}}} \\ [1em]
       & & \text{Stag} & \text{Hare} \\
        \begin{block}{cc(cc)}
\underline{\text{player 1}} & \text{Stag} & (10, 10) & (0, 6) \\
                            & \text{Hare} & (6, 0) & (6, 6) \\
        \end{block}
    \end{blockarray}
\end{equation*}

Suppose player 1 uses the mixed strategy \((x, 1- x)\), where \(x\) is player 1's
probability to Stag. Similarly, player 2's strategy is \((y, 1 - y)\).

\begin{enumerate}[(i)]
    \item For given \(x, y\) compute the players' payoffs \(\pi^{(1)}(x, y),
    \pi^{(2)}(x, y)\) (see Remarks 2.6, 2.7).
    \item For a given \(y\) compute player 1's best response (BR(\(y\))). In
    particular, show that there is some \(y^{*}\) such that all \(x \in [0,
    1]\) are a best response.
    \item Draw the two best response correspondences BR(\(x\)), BR(\(y\)) into a
    \(x-y\) plane. How often do they intersect? What does it mean if they
    intersect.
\end{enumerate}

\end{document}

