\documentclass[10pt]{article}

% Manage page layout
\usepackage[margin=2.5cm, includefoot, footskip=30pt]{geometry}
\pagestyle{plain}
\setlength{\parindent}{0em}
\setlength{\parskip}{1em}
\renewcommand{\baselinestretch}{1}

\usepackage{blkarray}
\usepackage{multirow}
\usepackage{amsmath}
\usepackage{enumerate}

\title{\textbf{Week 3.} Static games with complete information II: Nash equilibrium}
\date{}

\begin{document}
\maketitle
\vspace{-1cm}

\subsection*{Exercise 1: Iterated elimination of dominated strategies I}

\textbf{Construct a 2-player game such that:}

\begin{enumerate}
    \item Both players have 3 actions.
    \item The game cannot be solved by elimination of dominated strategies.
    \item The game can be solved by iterated elimination of dominated strategies.
\end{enumerate}

\subsection*{Exercise 2: Iterated elimination of dominated strategies II}

Consider the following 3-players game. Here the first player chooses a row,
the second player chooses a column, and the third player chooses a matrix.

\textbf{Can you solve this game using iterated elimination of strategies?}

\begin{equation*}
\begin{blockarray}{ccc}
    & \BAmulticolumn{2}{c}{\text{Matrix } 1} \\ [1em]
    & \text{Col } 1 & \text{Col } 2 \\
    \begin{block}{c(cc)}
        \text{Row } 1 & (2, 1, 6) & (3, 2, 3) \\
        \text{Row } 2 & (0, 4, 0) & (1, 0, 0) \\
    \end{block}
\end{blockarray}\qquad
%
\begin{blockarray}{ccc}
    & \BAmulticolumn{2}{c}{\text{Matrix } 2} \\ [1em]
    & \text{Col } 1 & \text{Col } 2 \\
    \begin{block}{c(cc)}
        \text{Row } 1 & (1, -1, 4) & (2, 1, 4) \\
        \text{Row } 2 & (-1, 4, 0) & (0, 0, 3) \\
    \end{block}
\end{blockarray}
\end{equation*}

\subsection*{Exercise 3: Nash equilibrium vs dominance solvability}

\textbf{Prove the following statements:}

\begin{enumerate}[(i)]
    \item If a pure strategy \(s^{(i)}_{j}\) is dominated by a pure strategy \(s^{(i)}_{k}\) and
    \(\sigma = (\sigma^{(1)}, \dots, \sigma^{(n)})\) is a Nash equilibrium, then
    \(\sigma^{(i)}_{j}=0\).
    \item If the game is dominance solvable such that the unique outcome of
    iterated elimination of dominated strategies is some pure strategy
    \(s=(s^{(1)}, \dots, s^{(n)})\), then \(s\) is a Nash equilibrium.
\end{enumerate}

[Suggestion: One could use contradiction to prove the above statements. For example, for (i)
assume that these was a Nash equilibrium with \(\sigma^{(i)}_{j}>0\), and show
that this would yield some contradiction.]

\subsection*{Exercise 4: Best responses}

Consider the stag hunt game:

\begin{equation*}
    \begin{blockarray}{cccc}
       & & \BAmulticolumn{2}{c}{\underline{\text{player 2}}} \\ [1em]
       & & \text{Stag} & \text{Hare} \\
        \begin{block}{cc(cc)}
\underline{\text{player 1}} & \text{Stag} & (10, 10) & (0, 6) \\
                            & \text{Hare} & (6, 0) & (6, 6) \\
        \end{block}
    \end{blockarray}
\end{equation*}

Suppose player 1 uses the mixed strategy \((x, 1- x)\), where \(x\) is player 1's
probability to Stag. Similarly, player 2's strategy is \((y, 1 - y)\).

\begin{enumerate}[(i)]
    \item For given \(x, y\) compute the players' payoffs \(\pi^{(1)}(x, y),
    \pi^{(2)}(x, y)\) (see Remarks 2.6, 2.7).
    \item For a given \(y\) compute player 1's best response (BR(\(y\))). In
    particular, show that there is some \(y^{*}\) such that all \(x \in [0,
    1]\) are a best response.
    \item Draw the two best response correspondences BR(\(x\)), BR(\(y\)) into a
    \(x-y\) plane. How often do they intersect? What does it mean if they
    intersect?
\end{enumerate}

\end{document}

