\documentclass[10pt]{article}

% Manage page layout
\usepackage[margin=2.5cm, includefoot, footskip=30pt]{geometry}
\pagestyle{plain}
\setlength{\parindent}{0em}
\setlength{\parskip}{1em}
\renewcommand{\baselinestretch}{1}

\usepackage{blkarray}
\usepackage{multirow}
\usepackage{amsmath}
\usepackage{enumerate}

\title{\textbf{Week 4.} Static games with complete information III: Nash equilibria}
\date{}

\begin{document}
\maketitle
\vspace{-1cm}

\subsection*{Exercise 1: Nash equilibrium vs dominance solvability}

\textbf{Prove the following statements:}

\begin{enumerate}[(i)]
    \item If a pure strategy \(s^{(i)}_{j}\) is dominated by a pure strategy \(s^{(i)}_{k}\) and
    \(\sigma = (\sigma^{(1)}, \dots, \sigma^{(n)})\) is a Nash equilibrium, then
    \(\sigma^{(i)}_{j}=0\).
    \item If the game is dominance solvable such that the unique outcome of
    iterated elimination of dominated strategies is some pure strategy
    \(s=(s^{(1)}, \dots, s^{(n)})\), then \(s\) is a Nash equilibrium.
\end{enumerate}

[Suggestion: One could use contradiction to prove the above statements. For example, for (i)
assume that these was a Nash equilibrium with \(\sigma^{(i)}_{j}>0\), and show
that this would yield some contradiction.]

\subsection*{Exercise 2: Best responses}

Consider the stag hunt game:

\begin{equation*}
    \begin{blockarray}{cccc}
       & & \BAmulticolumn{2}{c}{\underline{\text{player 2}}} \\ [1em]
       & & \text{Stag} & \text{Hare} \\
        \begin{block}{cc(cc)}
\underline{\text{player 1}} & \text{Stag} & (10, 10) & (0, 6) \\
                            & \text{Hare} & (6, 0) & (6, 6) \\
        \end{block}
    \end{blockarray}
\end{equation*}

Suppose player 1 uses the mixed strategy \((x, 1- x)\), where \(x\) is player 1's
probability to Stag. Similarly, player 2's strategy is \((y, 1 - y)\).

\begin{enumerate}[(i)]
    \item For given \(x, y\) compute the players' payoffs \(\pi^{(1)}(x, y),
    \pi^{(2)}(x, y)\) (see Remarks 2.6, 2.7).
    \item For a given \(y\) compute player 1's best response (BR(\(y\))). In
    particular, show that there is some \(y^{*}\) such that all \(x \in [0,
    1]\) are a best response.
    \item Draw the two best response correspondences BR(\(x\)), BR(\(y\)) into a
    \(x-y\) plane. How often do they intersect? What does it mean if they
    intersect?
\end{enumerate}

\subsection*{Exercise 3: Cournot Duopoly}

The Cournot duopoly game is defined by:

\begin{itemize}
    \item Players: \(N = \{\text{Firm } 1, \text{Firm } 2\}\)
    \item Actions: Amount of good produced, \(x^{(i)} \in [0, \infty)\) for \(i \in \{1, 2\}\)
    \item Payoffs: \(\pi^{(i)}(x^{(1)}, x^{(2)}) = [a - b (x^{(1)} + x^{(2)})] x^{(i)} - c x^{(i)}\)
\end{itemize}

\textbf{Show that there is a Nash equilibrium in pure strategies. For simplicity
assume \(a=10, b=1, c=1\)}.

[Hint: For each \(x^{(i)}\) computer BR\((x^{(-i)})\). Then solve simultaneously:
\begin{align*}
    x^{(1)}= \text{BR}(x^{(2)}) \\
    x^{(2)}= \text{BR}(x^{(1)})
\end{align*}
]

\subsection*{Exercise 4: Matching Pennies}

\textbf{Compute the Nash equilibria for the following two games, and interpret the result.}

\begin{equation*}
    \begin{blockarray}{ccc}
        & \text{Left} & \text{Right} \\
        \begin{block}{c(cc)}
            \text{Top} &    (0.8, 0.4) & (0.4, 0.8) \\
            \text{Bottom} & (0.4, 0.8) & (0.8, 0.4) \\
        \end{block}
    \end{blockarray}\qquad
    %
    \begin{blockarray}{ccc}
        & \text{Left} & \text{Right} \\
        \begin{block}{c(cc)}
            \text{Top} &    (3.2, 0.4) & (0.4, 0.8) \\
            \text{Bottom} & (0.4, 0.8) & (0.8, 0.4) \\
        \end{block}
    \end{blockarray}
    \end{equation*}

\subsection*{Bonus Exercise 1: Verifying NE in games with finitely many players \& actions}

Show that to verify whether a strategy profile \(\hat{\sigma} =
(\hat{\sigma}^{(1)}, \dots, \hat{\sigma}^{(n)})\) is a Nash equilibrium, it is
sufficient to check all deviations towards pure strategies.

\textbf{Specifically show that \(\hat{\sigma}\) is a Nash equilibrium if and only if for
all players \(i\) the following two conditions hold:}

\begin{enumerate}[(i)]
    \item All actions that player \(i\) uses give the same payoff: if
    \(\sigma^{(i)}_j > 0\) and \(\sigma^{(i)}_k > 0\) then
    \(\pi^{(i)}(s^{(i)}_j, \hat{\sigma}^{(-i)}) = \pi^{(i)}(s^{(i)}_k,
    \hat{\sigma}^{(-i)})\).
    \item Actions that are not played are not profitable: if \(\sigma^{(i)}_j =
    0\) then \(\pi^{(i)}(s^{(i)}_j, \hat{\sigma}^{(-i)}) \leq
    \pi^{(i)}(\hat{\sigma}^{(i)}, \hat{\sigma}^{(-i)})\).
\end{enumerate}

[Hint: One way to prove the above is once again by contradiction.]

\subsection*{Bonus Exercise 2: Finding games with a non-generic number of equilibria}

\textbf{Find an example of a symmetric 2 player game, with 2 actions per player, with:}

\begin{itemize}
    \item Exactly 2 Nash equilibria
    \item infinitely many Nash equilibria
\end{itemize}

[Note: These should include all Nash equilibria. Not just pure Nash equilibria.]

\end{document}

